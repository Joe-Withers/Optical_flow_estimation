% !TEX root = ./Unserpervised_flow_estimation.tex

\section{Implementation}



\subsection{unsupervised loss function}
Like in *refs*, we extend *FlowNet* to use a proxy flow error instead of using ground truth flows. The core of unsupervised flow estimation CNN's is a differentiable warping function *refs*. Then the photometric loss between the warped second image and the ground truth first image this can be considered our data term in our loss function. Using the photometric error in optical flow estimation follows from the brightness constancy assumption *ref*. In gradient and variation methods using only the brightness constancy results in an under-determined system. *start* However, training a neural network should be possible using only the photometric error. *end*. Using *eq horn and schunck*, we can replace the data term with our photometric error, giving *eq*. This is equivalent to adding a smoothness constraint, which can be considered a regularizing term.

\subsubsection{photometric loss}
differentiable warping function - reference to section\\
Charbonnier loss\\

\subsubsection{smoothness constraint}
first order smoothness constraint\\
second order smoothness constraint?\\

\subsubsection{occlusion error term?}

\subsection{network architecture}
diagrams\\
introduce how loss functions are used in the architecture.\\

\subsection{Differentiable warping}
bilinear interpolation\\

\subsubsection{Testing the warp function}
using ground truth flow and comparing error of warped image\\
